% !TEX TS-program = XeTeX
% !TEX encoding = UTF-8 Unicode
% !TEX spellcheck = gr-GR

\documentclass[12pt]{turabian-researchpaper}
\usepackage{fontspec}
\usepackage{polyglossia}
\usepackage{setspace}
\usepackage{ragged2e}
\usepackage{geometry} % to change the page dimensions
\usepackage{graphicx} % support the \includegraphics command and options
% \usepackage[indent]{parskip} % Activate to begin paragraphs with an empty line rather than an indent
\usepackage{booktabs} % for much better looking tables
\usepackage{array} % for better arrays (eg matrices) in maths
\usepackage{paralist} % very flexible & customisable lists (eg. enumerate/itemize, etc.)
\usepackage{fancyhdr}
\usepackage{multicol}
\usepackage{enumitem}
\usepackage[hidelinks]{hyperref}

\setlength{\headheight}{30pt}
\addtolength{\topmargin}{-15pt}

\geometry{a4paper} % or letterpaper (US) or a5paper or....
\geometry{margin=2.5cm} % for example, change the margins to 2
% inches all round
\setmainfont{Times New Roman}
\setdefaultlanguage{greek}

\pagestyle{fancy}
\fancyhf{}
\fancyhead[L]{\rightmark}
\fancyhead[R]{\thepage}
\rhead{\rightmark}
\lhead{Διαχείρηση Έργων Πληροφορικής}
\rfoot{\thepage}

% \setlist{12em}

% \setlength{\parindent}{0.5em}
% \setlength{\parskip}{0em}
\setcounter{tocdepth}{5}

\setstretch{1.2}
\justifying

\title{Εργασία 1}
\subtitle{Βιβλιογραφική επισκόπηση και συγκριτική αξιολόγηση μεθοδολογιών διαχείρισης έργων}
\author{Ζαμάγιας Μιχαήλ Ανάργυρος -- ΤΠ5000}
\course{Διαχείρηση Έργων Πληροφορικής}
\date{\today}

\begin{document}

\begin{titlepage}
    \maketitle
\end{titlepage}

\begin{abstract}
    Η παρούσα αναφορά πρόκειται για μία σύντομη και περιεκτική βιβλιογραφική επισκόπηση των μεθοδολογιών διαχείρισης έργων, \textbf{PMBOK},
    \textbf{PRINCE2} και \textbf{PM}\boldmath{$^2$}.
    Πιο συγκεκριμένα, ακολουθεί:
    \begin{enumerate}[itemsep=2pt]
        \item Περιγραφή των αρχών της εκάστοτε μεθοδολογίας.
        \item \textit{Συγκριτική αξιολόγηση} των προαναφερθέντων μεθοδολογιών.
    \end{enumerate}
\end{abstract}

\tableofcontents

\newpage

\section{Περιγραφή μεθοδολογιών}\label{methods}
\subsection{Η μεθοδολογία PMBOK}\label{method_pmbok}
Η μεθοδολογία PMBOK (Project Management Body Οf Knowledge), αποτελούμενη από διεργασίες, εργαλεία, τεχνικές και καλές πρακτικές, είναι ένα σύνολο γνώσης στο επιστημονικό πεδίο Διαχείριση Έργων Πληροφορικής. Είναι ένα πρότυπο διεθνώς αναγνωρισμένο, τόσο από το IEEE (Institute of Electrical and Electronics Engineers) όσο και από το ANSI (American National Standards Institute). Η PMBOK αναγνωρίζει 5 βασικές ομάδες διεργασιών (process groups) και 10 περιοχές γνώσης (knowledge areas) και οι αρχές αυτής εφαρμόζονται της μεθοδολογίας σε έργων (projects), ανάπτυξη λογισμικού (software development) και διάφορες άλλες λειτουργίες (operations). \par
{\parindent0pt
    Οι 5 ομάδες διεργασιών:
    \begin{enumerate}[itemsep=0pt]
        \item Έναρξης (Initiating)
        \item Προγραμματισμού (Planning)
        \item Εκτέλεσης (Execution)
        \item Ελέγχου (Controlling)
        \item Τερματισμού (Closing)
    \end{enumerate}
    Οι 10 περιοχές γνώσης (PMI, 2013):
    \begin{enumerate}[itemsep=0pt]
        \item Ενοποίηση του έργου (Project integration management)
        \item Διαχείριση αντικειμένου εργασιών έργου (Project scope management)
        \item Διαχείριση χρόνου έργου (Project time management)
        \item Διαχείριση κόστους έργου (Project cost management)
        \item Διαχείριση ποιότητας έργου (Project quality management)
        \item Διαχείριση ανθρωπίνων πόρων έργου (Project human resource management)
        \item Διαχείριση επικοινωνίας έργου (Project communication management)
        \item Διαχείριση κινδύνων έργου (Project risk management)
        \item Διαχείριση προμηθειών έργου (Project procurement management)
        \item Διαχείριση συμμετεχόντων (Stakeholder management)
    \end{enumerate}
}

\subsection{Η μεθοδολογία PRINCE2}\label{method_prince2}

\subsection{Η μεθοδολογία PM\texorpdfstring{$^2$}{}}\label{method_pm2}

\newpage\section{Συγκριτική αξιολόγηση μεθοδολογιών}\label{comparisons}
\subsection{Η μεθοδολογία PMBOK}\label{comparison_pmbok}
Οι χρήστες αυτού του συστήματος διαπιστώνουν ότι έχει πιο ουσιαστικά πλαίσια για τη διαχείριση συμβάσεων, τη διαχείριση του πεδίου και άλλα χαρακτηριστικά που είναι αναμφισβήτητα λιγότερο σταθερά στο PRINCE2. Ωστόσο, πολλοί χρήστες του PMBOK θεωρούν ότι δεν είναι απολύτως ευχαριστημένοι με τον τρόπο που αυτό το σύστημα περιορίζει τη λήψη αποφάσεων αποκλειστικά στους διαχειριστές έργων, καθιστώντας δύσκολη την παράδοση κομματιών της διαχείρισης σε άλλα κλιμάκια και ανώτερους διευθυντές. Με το PMBOK, ο διαχειριστής του έργου μπορεί φαινομενικά να γίνει ο κύριος υπεύθυνος λήψης αποφάσεων, αρμόδιος για το σχεδιασμό, επίλυση προβλημάτων, διαχειριστής ανθρώπινου δυναμικού και ούτω καθεξής.

\subsection{Η μεθοδολογία PRINCE2}\label{comparison_prince2}
Πρόκειται για ένα πρόγραμμα διαχείρισης έργων που μοιράζει περισσότερο τη λειτουργική και οικονομική αρχή με τα ανώτερα στελέχη, όχι αποκλειστικά στον διαχειριστή του έργου. Αυτό το πρόγραμμα επικεντρώνεται στη βοήθεια του διαχειριστή του έργου για την επίβλεψη έργων για λογαριασμό της ανώτερης διοίκησης ενός οργανισμού. Από την πλευρά των πλεονεκτημάτων, το PRINCE2 παρέχει μια ενιαία στάνταρ προσέγγιση στην διαχείριση έργων, γι 'αυτό πολλοί κυβερνητικοί και παγκόσμιοι οργανισμοί προτιμούν αυτήν την επιλογή. Προτιμάται επίσης λόγω της ευκολίας χρήσης του, η οποία είναι εύκολη στην εκμάθηση, ακόμη και για άτομα με περιορισμένη εμπειρία. Στο μειονέκτημα, υπάρχουν χρήστες που πιστεύουν ότι το PRINCE2 χάνει τη σημασία των «μαλακών δεξιοτήτων» που πρέπει να είναι το επίκεντρο ενός διαχειριστή έργου.

\subsection{Η μεθοδολογία PM\texorpdfstring{$^2$}{}}\label{comparison_pm2}

\newpage\section{Βιβλιογραφία}
\subsection{\texorpdfstring{\hyperref[methods]{Περιγραφή μεθοδολογιών}}{}}
\subsubsection{\texorpdfstring{\hyperref[method_pmbok]{Η μεθοδολογία PMBOK}}{}}

\begin{itemize}[itemsep=0pt]
    \item \href{https://en.wikipedia.org/wiki/Project_Management_Body_of_Knowledge}{Wikipedia}
    \item \href{https://repository.kallipos.gr/bitstream/11419/2262/3/06_kefalaio6.pdf}{Αποθετήριο «Κάλλιπος»}
    \item \href{https://dione.lib.unipi.gr/xmlui/bitstream/handle/unipi/8488/Fragkos_Georgios.pdf}{Βιβλιοθήκη Πανεπιστημίου Πειραιώς «Διώνη»}
    \item \href{https://www.slideshare.net/CTESolutions/prince2-pmbok-comparison-demystified-29846454}{PRINCE2 PMBOK Comparison Demystified}
\end{itemize}

\subsubsection{\texorpdfstring{\hyperref[method_prince2]{Η μεθοδολογία PRINCE2}}{}}

\subsubsection{\texorpdfstring{\hyperref[method_pm2]{Η μεθοδολογία PM$^2$}}{}}

\subsection{\texorpdfstring{\hyperref[comparisons]{Συγκριτική αξιολόγηση μεθοδολογιών}}{}}

\subsubsection{\texorpdfstring{\hyperref[comparison_pmbok]{Η μεθοδολογία PMBOK}}{}}
\begin{itemize}[itemsep=0pt]
    \item \href{https://www2.cio.com.au/article/402347/pmbok_vs_prince2_vs_agile_project_management/}{PMBOK vs PRINCE2 vs Agile project management}
\end{itemize}

\subsubsection{\texorpdfstring{\hyperref[comparison_prince2]{Η μεθοδολογία PRINCE2}}{}}
\begin{itemize}[itemsep=0pt]
    \item \href{https://www2.cio.com.au/article/402347/pmbok_vs_prince2_vs_agile_project_management/}{PMBOK vs PRINCE2 vs Agile project management}
\end{itemize}

\subsubsection{\texorpdfstring{\hyperref[comparison_pm2]{Η μεθοδολογία PM\texorpdfstring{$^2$}{}}}{}}

\end{document}
