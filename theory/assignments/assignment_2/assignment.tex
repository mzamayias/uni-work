% !TEX TS-program = XeTeX
% !TEX encoding = UTF-8 Unicode
% !TEX spellcheck = gr-GR

\documentclass[12pt]{turabian-researchpaper}
\usepackage{fontspec}
\usepackage{polyglossia}
\usepackage{setspace}
\usepackage{ragged2e}
\usepackage{geometry}
\usepackage{graphicx}
\usepackage{titlesec}
\usepackage{booktabs}
\usepackage{array}
\usepackage{amsmath}
\usepackage{paralist}
\usepackage{fancyhdr}
\usepackage{multicol}
\usepackage{pagecolor}
\usepackage{enumitem}
\usepackage{svg}
\usepackage{xcolor}
\usepackage[hidelinks]{hyperref}

\geometry{a4paper}
\geometry{margin=2.4cm}
\setmainfont{GFS Didot}
\setdefaultlanguage{greek}

\setstretch{1.2}
\justifying

\definecolor{calm white}{HTML}{F3F3F3}
\definecolor{dark grey}{HTML}{303030}
\pagecolor{calm white}
\color{dark grey}

\graphicspath{ {.} }

\titleformat{\section}
       {\normalfont\fontsize{15}{18}\bfseries}
       {\thesection}
       {1em}
       {}
\titleformat{\subsection}
       {\normalfont\fontsize{12}{15}\bfseries\itshape}
       {\thesubsection}
       {1em}
       {}

\title{Εργασία 2}
\subtitle{Ασκήσεις CPM, PERT και Resource Management}
\author{Ζαμάγιας Μιχαήλ Ανάργυρος -- ΤΠ5000}
\course{Διαχείρηση Έργων Πληροφορικής}
\date{\today}

\begin{document}

\begin{titlepage}
    \maketitle
\end{titlepage}

\begin{abstract}
    Η παρούσα αναφορά πρόκειται για την παράθεση των λύσεων μου προς τις ασκήσεις της εκφώνησης.
\end{abstract}

\tableofcontents

\newpage

\section{Άσκηση CPM}
Μια εταιρεία έχει αναλάβει την ανάπτυξη ενός μεγάλου πληροφοριακού συστήματος. Το όλο έργο απαιτεί για την ολοκλήρωσή του την υλοποίηση 12 δραστηριοτήτων. Οι σχέσεις μεταξύ των δραστηριοτήτων καθώς και οι διάρκειες δίνονται στον παρακάτω πίνακα:
\begin{table}
    \centering
    \begin{tabular}{lll}
        \hline
        Προγενέστερη Δραστηριότητα & Διάρκεια (σε ημέρες) & Δραστηριότητα \\ \hline
        100                        & 14                   & 201, 202      \\
        101                        & 10                   & 301           \\
        102                        & 13                   & 203, 204      \\
        201                        & 15                   & 302           \\
        202                        & 25                   & 301           \\
        203                        & 12                   & 301           \\
        204                        & 14                   & 303, 304      \\
        301                        & 17                   & 401           \\
        302                        & 13                   & 401           \\
        303                        & 10                   & 401           \\
        304                        & 10                   & -             \\
        401                        & 11                   & -             \\ \hline
    \end{tabular}
\end{table}

\subsection{Ποια είναι η κρίσιμη διαδρομή και ποιος ο χρόνος περάτωσης του έργου;}
\begin{figure}
    \centering
    \def\svgwidth{\columnwidth}
    \input{image.pdf_tex}
\end{figure}
Η κρίσιμη διαδρομή είναι η \{100, 202, 301, 401, END\} και ο χρόνος περάτωσης του έργου είναι 67 ημέρες.

\subsection{Αν η δραστηριότητα 203 καθυστερήσει κατά 9 μέρες θα επηρεαστεί ο χρόνος υλοποίησης του έργου και γιατί;}
Δεν θα επηρεαστεί ο χρόνος περάτωσης του έργου αν η δραστηριότητα 203 καθυστερήσει κατά 9 μέρες, επειδή το Slack της δραστηριότητας 203 είναι μεγαλύτερο της διάρκειάς της. Επίσης, δεν αλλάζει η κρίσιμη διαδρομή καθώς το νέο $SL_{203}$ παραμένει διάφορο του μηδενός:
\begin{center}
    $ SL_{203} = 14 $ και $ D = 9 $
\end{center}
$$ SL_{203}' = SL_{203} - D $$
$$ SL_{203}' = 14 - 9 $$
$$ SL_{203}' = 5 $$
$$ SL_{203}' \neq 0 $$

\subsection{Αν η δραστηριότητα 204 καθυστερήσει κατά 19 μέρες αντί για 14, τι θα συμβέι σε σχέση με τον χρόνο υλοποίησης του έργου και γιατί;}
Δεν θα επηρεαστεί ο χρόνος περάτωσης του έργου αν η δραστηριότητα 204 καθυστερήσει 19 μέρες αντί 14, επειδή το Slack της δραστηριότητας 204 είναι μεγαλύτερο της διάρκειάς της. Επίσης, δεν αλλάζει η κρίσιμη διαδρομή καθώς το νέο $SL_{204}$ παραμένει διάφορο του μηδενός:
\begin{center}
    $ SL_{204} = 19 $ και $ D = 19 - 14 = 5 $
\end{center}
$$ SL_{204}' = SL_{204} - D $$
$$ SL_{204}' = 19 - 5 $$
$$ SL_{204}' = 14 $$
$$ SL_{204}' \neq 0 $$

\subsection{Ποια είναι η νέα κρίσιμη διαδρομή και ποιος ο νέος χρόνος περάτωσης του έργου;}
\textit{\textbf{Δεδομένου ότι, σε σύσκεψη που έγινε μετά την πάροδο 16 ημερών από την έναρξη του έργου, διαπιστώθηκαν τα ακόλουθα:}}
\begin{itemize}
    \item \textit{\textbf{Οι δραστηριότητες 100, 101 και 102 είχαν πραγματοποιηθεί σύμφωνα με τον αρχικό προγραμματισμό.}}
    \item \textit{\textbf{Οι δραστηριότητες 201 κι 202 είναι σε εξέλιξη και απαιτούν 6 και 10 ημέρες αντίστοιχα για να ολοκληρωθούν.}}
    \item \textit{\textbf{Οι δραστηριότητες 203 και 204 είναι επίσης σε εξέλιξη και απαιτούν 3 και 21 ημέρες αντίστοιχα για να ολοκληρωθούν.}}
    \item \textit{\textbf{Για την δραστηριότητα 303 έγινε νέα εκτίμηση της διάρκειάς της και υπολογίστηκε ότι απαιτεί 12 ημέρες για να ολοκληρωθεί.}}
    \item \textit{\textbf{Για την δραστηριότητα 304 αποφασίστηκε ότι μπορεί να υλοποιηθεί σε 5 ημέρες, ενώ οι υπόλοιπες δραστηριότητες υπολογίζεται ότι θα εκτελεστούν σύμφωνα με τον αρχικό προγραμματισμό.}}
\end{itemize}

\section{Άσκηση PERT}
Η φάση σχεδίασης -- κωδικοποίησης ενός μικρού πακέτου λογισμικού εκτιμάται ότι περιέχει αστάθμητους παράγοντες. Γι' αυτό το λόγο οι εννέα δραστηριότητες που την αποτελούν εκτιμήθηκαν με 3 διαφορετικές διάρκειες: την αισιόδοξη (a), την πιο πιθανή (m) και την απαισιόδοξη (b).
Οι εκτιμήσεις για τις διάρκειες (\textbf{σε εβδομάδες}) δίδονται στον ακόλουθο πίνακα:
\begin{table}[]
    \centering
    \begin{tabular}{llll}
        \hline
        Δραστηριότητα & a   & m & b   \\ \hline
        \textbf{A}    & 0.5 & 1 & 1.5 \\
        \textbf{B}    & 2   & 4 & 6   \\
        C             & 2   & 3 & 4   \\
        D             & 6   & 7 & 8   \\
        \textbf{E}    & 4   & 6 & 8   \\
        F             & 1   & 2 & 3   \\
        \textbf{G}    & 6   & 7 & 8   \\
        H             & 6   & 9 & 12  \\
        \textbf{I}    & 2   & 4 & 6   \\ \hline
    \end{tabular}
\end{table}\newline
Έστω πως το κρίσιμο μονοπάτι περιλαμβάνει τις δραστηριότητες \textbf{A, B, E, G, I}, απαντήστε στις ακόλουθες ερωτήσεις:

\subsection{Ποιος είναι ο αναμενόμενος χρόνος ολοκλήρωσης του έργου;}
\begin{equation*}
    \begin{aligned}
         & t_{exp(A)} = \frac{t_a+4*t_m+t_b}{6} = \frac{0.5+4*1+1.5}{6} = 1                 \\
         & t_{exp(B)} = \frac{t_a+4*t_m+t_b}{6} = \frac{2+4*4+6}{6} = 4                     \\
         & t_{exp(E)} = \frac{t_a+4*t_m+t_b}{6} = \frac{4+4*6+8}{6} = 6                     \\
         & t_{exp(G)} = \frac{t_a+4*t_m+t_b}{6} = \frac{6+4*7+8}{6} = 7                     \\
         & t_{exp(I)} = \frac{t_a+4*t_m+t_b}{6} = \frac{2+4*4+6}{6} = 4                     \\
         & t_{exp}    = t_{exp(A)} + t_{exp(B)} +t_{exp(E)} +t_{exp(G)}+    t_{exp(I)} = 22
    \end{aligned}
\end{equation*}
Ο αναμενόμενος χρόνος ολοκλήρωσης του έργου είναι 22 εβδομάδες.

\subsection{Ποια η πιθανότητα το έργο να ολοκληρωθεί μια εβδομάδα πιο πριν από ότι αναμένεται;}
\begin{equation*}
    \begin{aligned}
         & var_A = \frac{(1.5-0.5)^2}{6^2} = \frac{1}{36}        \\
         & var_B = \frac{(6-2)^2}{6^2} = \frac{16}{36}           \\
         & var_E = \frac{(8-4)^2}{6^2} = \frac{16}{36}           \\
         & var_G = \frac{(8-6)^2}{6^2} = \frac{4}{36}            \\
         & var_I = \frac{(6-2)^2}{6^2} = \frac{16}{36}           \\
         & var   = var_A + var_B + var_E + var_G + var_I = 1.472 \\
         & \sqrt{var} = 1.21326                                  \\
         & z = \frac{21-22}{1.21326} = -0.82                     \\
    \end{aligned}
\end{equation*}
Για $ z = -0.82 $, το έργο έχει πιθανότητα $ 20.61 \% $ να τελειώσει μια εβδομάδα νωρίτερα.

\subsection{Ποια η πιθανότητα το έργο να μην ολοκληρωθεί εντός 24 εβδομάδων;}
\begin{equation*}
    \begin{aligned}
         & P(x \geq 24) = 1 - P(x \leq 24)                                   \\
         & z = \frac{24-22}{1.21326} = 1.6484 \implies P(x \leq 24) = 0.9505 \\
         & P(x \geq 24) = 1 - 0.9505 = 0.0495                                \\
    \end{aligned}
\end{equation*}
Η πιθανότητα το έργο να μην ολοκληρωθεί εντός 24 εβδομάδων είναι $ 4.95 \% $.

\subsection{Αν θέλουμε να έχουμε πιθανότητα μόνο 10\% να αποτύχουμε στον προγραμματισμό των ενεργειών μας, τότε πόσο εκτιμάτε πως θα διαρκέσει το έργο;}
\begin{equation*}
    \begin{aligned}
         & 10\% \text{ πιθανότητα για αποτυχία} \implies 90\% \text{ πιθανότητα για επιτυχία} \\
         & z = 1.29 \implies 1.29 = \frac{x - 22}{1.21326} \implies 1.56510 = x - 22 \implies x = 23.5651
    \end{aligned}
\end{equation*}
Το έργο θα έχει τελειώσει στις $ 23.5 $ εβδομάδες με πιθανότητα $ 90 \% $.

\end{document}
