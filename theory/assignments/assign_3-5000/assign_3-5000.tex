% !TEX TS-program = XeTeX
% !TEX encoding = UTF-8 Unicode
% !TEX spellcheck = gr-GR

\documentclass[12pt]{turabian-researchpaper}
\usepackage{fontspec}
\usepackage{polyglossia}
% \usepackage{setspace}
\usepackage{ragged2e}
\usepackage{amsmath}
\usepackage{geometry} % to change the page dimensions
% \usepackage{graphicx} % support the \includegraphics command and optionsv
% \usepackage[indent]{parskip} % Activate to begin paragraphs with an empty line rather than an indent
% \usepackage{booktabs} % for much better looking tables
% \usepackage{array} % for better arrays (eg matrices) in maths
% \usepackage{paralist} % very flexible & customisable lists (eg. enumerate/itemize, etc.)
\usepackage{fancyhdr}
% \usepackage{multicol}
% \usepackage{enumitem}
\usepackage[hidelinks]{hyperref}

\setlength{\headheight}{30pt}
\addtolength{\topmargin}{-15pt}

\geometry{a4paper} % or letterpaper (US) or a5paper or....
\geometry{margin=2.5cm} % for example, change the margins to 2.5cm all round
\setmainfont{GFS Didot}
\setdefaultlanguage{greek}

\pagestyle{fancy}
\fancyhf{}
\fancyhead[L]{\chaptername}
\fancyhead[R]{\thepage}
\rhead{\chaptername}
\lhead{Ψηφιακή Επεξεργασία Σημάτων}
\rfoot{\thepage}

% \setlist{12em}

% \setlength{\parindent}{0.5em}
% \setlength{\parskip}{0em}
\setcounter{tocdepth}{5}

\setstretch{1.2}
\justifying

\title{Θεωρία}
\subtitle{Άσκηση 3}
\author{Ζαμάγιας Μιχαήλ Ανάργυρος -- ΤΠ5000}
\course{Ψηφιακή Επεξεργασία Σημάτων}
\date{\today}

\begin{document}

\begin{titlepage}
    \maketitle
\end{titlepage}

\tableofcontents

\newpage\section{Άσκηση 1}
Να βρεθεί ο μετασχηματισμός Ζ κάθε μιας από τις παρακάτω ακολουθίες:


Ακολουθία A
\begin{equation*}
    \begin{aligned}
        x(n) = (0.5)^n u(n) + (-0.6)^n u(n)
    \end{aligned}
\end{equation*}

Ακολουθία B
\begin{equation*}
    \begin{aligned}
        x(n) = \begin{cases}
            (0.3)^{| n |} \text{, } & | n | < 4     \\
            0\text{, }              & \text{αλλού.} \\
        \end{cases}
    \end{aligned}
\end{equation*}

\section{Άσκηση 2}

Για το παρακάτω σύστημα:

$$ y(n) = 0.8y(n-1) - 0.52y(n-2) + x(n) + 0.2x(n-1) - 0.15x(n-2) $$

Α. Να βρεθεί η συνάρτηση μεταφοράς του συστήματος.

Β. Να βρεθεί η κρουστική απόκριση του συστήματος με την μέθοδο άθροισμα μερικών κλασμάτων.

Γ. Να βρεθεί η έξοδος στο σήμα $ x(n) = u(n) - u(n-5) $.

\section{Άσκηση 3}
Για το παρακάτω γραμμικό, διακριτό, αιτιατό,  χρονικά αμετάβλητο κατά την μετατόπιση σύστημα με μηδέν και πόλους στα σημεία:

\begin{center}
    $ p_1 = 0.4 + j0.6 $, $ p_2 = 0.4 - j0.5 $, $ z_1 = 0.5 $, $ z_2 = -0.6 $
\end{center}

A. Υπολογίστε την συνάρτηση μεταφοράς του συστήματος και να σχεδιάσετε τα μηδέν και τους πόλους στο z επίπεδο.

Β. Υπολογίστε την κρουστική απόκριση του συστήματος $h(n)$ χρησιμοποιώντας  την μέθοδο υπολοίπων.

Γ. To σύστημα είναι σταθερό και γιατί;

\section{Άσκηση 4}

Για το παρακάτω IIR φίλτρο με μηδέν και πόλους στα σημεία:

\begin{center}
    $ p_1 = 0.4 + j0.6 $, $ p_2 = 0.4 - j0.5 $, $ z_1 = 0.5 $, $ z_2 = -0.6 $
\end{center}

Α. Σχεδιάστε τους πόλους και τα μηδέν στον μοναδιαίο κύκλο.

Β. Να βρεθεί το μέγεθος και η φάση του φίλτρου χρησιμοποιώντας γεωμετρική εκτίμηση (geometric evaluation) στις συχνότητες $\pi/4$, $\pi/2$, $ 3\pi/4 $, $ \pi $.

\section{Άσκηση 5}

Να βρεθεί με τις τρεις μεθόδους ο αντίστροφος μετασχηματισμός Z της συνάρτησης: $$ H(z) = \frac{3 z^2 + 0.4 z + 1}{(z + 0.2)(z - 0.5)} $$

\begin{equation}
    \begin{aligned}
        H(z) & = \frac{3 z^2 + 0.4 z + 1}{(z + 0.2)(z - 0.5)} \\
             & = \frac{3 z^2 + 0.4 z + 1}{z^2 - 0.3 z - 0.1}  \\
    \end{aligned}
\end{equation}

\end{document}
