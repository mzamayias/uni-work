% !TEX TS-program = XeTeX
% !TEX encoding = UTF-8 Unicode
% !TEX spellcheck = gr-GR

\documentclass[12pt]{turabian-researchpaper}
\usepackage{fontspec}
\usepackage{polyglossia}
% \usepackage{setspace}
\usepackage{ragged2e}
\usepackage{amsmath}
\usepackage{geometry} % to change the page dimensions
% \usepackage{graphicx} % support the \includegraphics command and optionsv
% \usepackage[indent]{parskip} % Activate to begin paragraphs with an empty line rather than an indent
% \usepackage{booktabs} % for much better looking tables
% \usepackage{array} % for better arrays (eg matrices) in maths
% \usepackage{paralist} % very flexible & customisable lists (eg. enumerate/itemize, etc.)
\usepackage{fancyhdr}
% \usepackage{multicol}
% \usepackage{enumitem}
\usepackage[hidelinks]{hyperref}

\setlength{\headheight}{30pt}
\addtolength{\topmargin}{-15pt}

\geometry{a4paper} % or letterpaper (US) or a5paper or....
\geometry{margin=2.5cm} % for example, change the margins to 2.5cm all round
\setmainfont{GFS Didot}
\setdefaultlanguage{greek}

\pagestyle{fancy}
\fancyhf{}
\fancyhead[L]{\rightmark}
\fancyhead[R]{\thepage}
\rhead{\rightmark}
\lhead{Ψηφιακή Επεξεργασία Σημάτων}
\rfoot{\thepage}

% \setlist{12em}

% \setlength{\parindent}{0.5em}
% \setlength{\parskip}{0em}
\setcounter{tocdepth}{5}

\setstretch{1.2}
\justifying

\title{Εργασία 3}
% \subtitle{Συνάρτηση Μεταφοράς και Περιοχή Σύγκλισης}
\author{Ζαμάγιας Μιχαήλ Ανάργυρος -- ΤΠ5000}
\course{Ψηφιακή Επεξεργασία Σημάτων}
\date{\today}

\begin{document}

\begin{titlepage}
    \maketitle
\end{titlepage}

\tableofcontents

\newpage\section{Άσκηση 1}
Να βρεθεί ο μετασχηματισμός Ζ κάθε μιας από τις παρακάτω ακολουθίες:
\subsection{Ακολουθία A}
\begin{equation*}
    \begin{aligned}
        x(n) = (0.5)^n u(n) + (-0.6)^n u(n)
    \end{aligned}
\end{equation*}

\subsection{Ακολουθία B}
\begin{equation*}
    \begin{aligned}
        x(n) = (0.3)^{| n |} \text{, } | n | < 4
    \end{aligned}
\end{equation*}

\begin{equation*}
    \begin{aligned}
        X(z) & = \sum_{n = 4}^{ 4 } x(n)z^{-n} \implies                                                                           \\
             & = \sum_{n = 4}^{ 4 } 0.3^{|n|} z^{-n} \implies                                                                     \\
             & = 0.3^{-3} z^{3} + 0.3^{-2} z^{2} + 0.3^{-1} z^{1} + 1 + 0.3^{1} z^{-1} + 0.3^{2} z^{-2} + 0.3^{3} z^{-3} \implies \\
    \end{aligned}
\end{equation*}


\newpage\section{Άσκηση 2}



\end{document}
