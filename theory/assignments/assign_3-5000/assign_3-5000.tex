% !TEX TS-program = XeTeX
% !TEX encoding = UTF-8 Unicode
% !TEX spellcheck = gr-GR

\documentclass[12pt]{turabian-researchpaper}
\usepackage{fontspec}
\usepackage{polyglossia}
% \usepackage{setspace}
\usepackage{ragged2e}
\usepackage{amsmath}
\usepackage{geometry} % to change the page dimensions
% \usepackage{graphicx} % support the \includegraphics command and optionsv
% \usepackage[indent]{parskip} % Activate to begin paragraphs with an empty line rather than an indent
% \usepackage{booktabs} % for much better looking tables
% \usepackage{array} % for better arrays (eg matrices) in maths
% \usepackage{paralist} % very flexible & customisable lists (eg. enumerate/itemize, etc.)
\usepackage{fancyhdr}
% \usepackage{multicol}
% \usepackage{enumitem}
\usepackage[hidelinks]{hyperref}

\setlength{\headheight}{30pt}
\addtolength{\topmargin}{-15pt}

\geometry{a4paper} % or letterpaper (US) or a5paper or....
\geometry{margin=2.5cm} % for example, change the margins to 2.5cm all round
\setmainfont{GFS Didot}
\setdefaultlanguage{greek}

\pagestyle{fancy}
\fancyhf{}
\fancyhead[L]{\leftmark}
\fancyhead[R]{\thepage}
\rhead{\leftmark}
\lhead{Ψηφιακή Επεξεργασία Σημάτων}
\rfoot{\thepage}

% \setlist{12em}

% \setlength{\parindent}{0.5em}
% \setlength{\parskip}{0em}
\setcounter{tocdepth}{5}

\setstretch{1.2}
\justifying

\title{Θεωρία}
\subtitle{Άσκηση 3}
\author{Ζαμάγιας Μιχαήλ Ανάργυρος -- ΤΠ5000}
\course{Ψηφιακή Επεξεργασία Σημάτων}
\date{\today}

\begin{document}

\begin{titlepage}
    \maketitle
\end{titlepage}

\tableofcontents

\newpage\section{Άσκηση 1}
Να βρεθεί ο μετασχηματισμός Ζ κάθε μιας από τις παρακάτω ακολουθίες:
\subsection{Ακολουθία A}
\begin{equation*}
    \begin{aligned}
        x(n) = (0.5)^n u(n) + (-0.6)^n u(n)
    \end{aligned}
\end{equation*}

\subsection{Ακολουθία B}
\begin{equation*}
    \begin{aligned}
        x(n) = \begin{cases}
            (0.3)^{| n |} \text{, } & | n | < 4     \\
            0\text{, }              & \text{αλλού.} \\
        \end{cases}
    \end{aligned}
\end{equation*}

Για $ 0 \leq n < 4 $:
\begin{equation*}
    \begin{aligned}
        X(z) & = \sum_{n = 0}^{ 3 } x(n)z^{-n}                                          \implies \\
             & = \sum_{n = 0}^{ 3 } 0.3^{|n|} z^{-n}                                    \implies \\
             & = \sum_{n = 0}^{ 3 } 0.3^{n} z^{-n}                                      \implies \\
             & = \sum_{n = 0}^{ 3 } (0.3 z^{ -1 })^n                                    \implies \\
             & = \sum_{n = 0}^{ 3 } ( \frac{ 0.3 }{ z } )^n                             \implies \\
             & = \frac{ 1 - ( \frac{ 0.3 }{ z } )^{ \infty } }{ 1 - \frac{ 0.3 }{ z } } \implies \\
             & = \frac{ 1 }{ 1 - 0.3 z^{ -1 } }                                         \implies \\
             & = \frac{ z }{ z - 0.3 }
    \end{aligned}
\end{equation*}

Για $ 4 < n < 0 $:
\begin{equation*}
    \begin{aligned}
        X(z) & = \sum_{n = -3}^{ -1 } x(n)z^{-n}              \implies \\
             & = \sum_{n = -3}^{ -1 } 0.3^{|n|} z^{-n}        \implies \\
             & = \sum_{n = -3}^{ -1 } 0.3^{-n} z^{-n}         \implies \\
             & = \sum_{n = -3}^{ -1 } (0.3 z)^{ -n }          \implies \\
             & = \sum_{n = -3}^{ -1 } \frac{ 1 }{ (0.3 z)^n } \implies \\
             & = 0.027 z^3 + 0.09 z^2 + 0.3 z
    \end{aligned}
\end{equation*}


\newpage\section{Άσκηση 2}

Για το παρακάτω σύστημα: $$ y(n) = 0.8y(n-1) - 0.52y(n-2) + x(n) + 0.2x(n-1) - 0.15x(n-2) $$
\subsection{Α. Να βρεθεί η συνάρτηση μεταφοράς του συστήματος.}
\subsection{Β. Να βρεθεί η κρουστική απόκριση του συστήματος με την μέθοδο άθροισμα μερικών κλασμάτων.}
\subsection{Γ. Να βρεθεί η έξοδος στο σήμα}
$x(n)=u(n)-u(n-5)$

\newpage\section{Άσκηση 3}
Για το παρακάτω γραμμικό, διακριτό, αιτιατό,  χρονικά αμετάβλητο κατά την μετατόπιση σύστημα με μηδέν και πόλους στα σημεία:

$p_1 = 0.4 + j0.6$, $p_2 = 0.4 - j0.5$, $z_1 = 0.5 $, $z_2 = -0.6$

\subsection{A. Υπολογίστε την συνάρτηση μεταφοράς του συστήματος και να σχεδιάσετε τα μηδέν και τους πόλους στο z επίπεδο.}
\subsection{Β. Υπολογίστε την κρουστική απόκριση του συστήματος \texorpdfstring{$h(n)$}{} χρησιμοποιώντας  την μέθοδο υπολοίπων.}
\subsection{Γ. To σύστημα είναι σταθερό και γιατί;}

\newpage\section{Άσκηση 4}

Για το παρακάτω IIR φίλτρο με μηδέν και πόλους στα σημεία:

$p_1 = 0.4 + j0.6$, $p_2 = 0.4 - j0.5$, $z_1 = 0.5 $, $z_2 = -0.6$

\subsection{Α. Σχεδιάστε τους πόλους και τα μηδέν στον μοναδιαίο κύκλο.}
\subsection{Β. Να βρεθεί το μέγεθος και η φάση του φίλτρου χρησιμοποιώντας γεωμετρική εκτίμηση (geometric evaluation) στις συχνότητες \texorpdfstring{$π/4$}{},  \texorpdfstring{$π/2$}{}, \texorpdfstring{$3π/4$}{}, \texorpdfstring{$π$}{}.}

\newpage\section{Άσκηση 5}

Να βρεθεί ο αντίστροφος μετασχηματισμός Z της συνάρτησης:

$$ H(z) = \frac{3z^2+0.4z+1}{(z+0.2)(z-0.5)}$$

Με τις τρεις μεθόδους.



\end{document}
