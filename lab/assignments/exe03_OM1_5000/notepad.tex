% !TEX TS-program = XeTeX
% !TEX encoding = UTF-8 Unicode
% !TEX spellcheck = gr-GR

\documentclass[12pt]{turabian-researchpaper}
\usepackage{fontspec}
\usepackage{polyglossia}
% \usepackage{setspace}
\usepackage{ragged2e}
\usepackage{amsmath}
\usepackage{geometry} % to change the page dimensions
% \usepackage{graphicx} % support the \includegraphics command and options
% \usepackage[indent]{parskip} % Activate to begin paragraphs with an empty line rather than an indent
% \usepackage{booktabs} % for much better looking tables
% \usepackage{array} % for better arrays (eg matrices) in maths
% \usepackage{paralist} % very flexible & customisable lists (eg. enumerate/itemize, etc.)
\usepackage{fancyhdr}
% \usepackage{multicol}
% \usepackage{enumitem}
\usepackage[hidelinks]{hyperref}

\setlength{\headheight}{30pt}
\addtolength{\topmargin}{-15pt}

\geometry{a4paper} % or letterpaper (US) or a5paper or....
\geometry{margin=2.5cm} % for example, change the margins to 2.5cm all round
\setmainfont{GFS Didot}
\setdefaultlanguage{greek}

\pagestyle{fancy}
\fancyhf{}
\fancyhead[L]{\rightmark}
\fancyhead[R]{\thepage}
\rhead{\rightmark}
\lhead{Ψηφιακή Επεξεργασία Σημάτων}
\rfoot{\thepage}

% \setlist{12em}

% \setlength{\parindent}{0.5em}
% \setlength{\parskip}{0em}
\setcounter{tocdepth}{5}

\setstretch{1.2}
\justifying

\title{Εργασία 3}
\subtitle{Συνάρτηση Μεταφοράς και Περιοχή Σύγκλισης}
\author{Ζαμάγιας Μιχαήλ Ανάργυρος -- ΤΠ5000}
\course{Ψηφιακή Επεξεργασία Σημάτων}
\date{\today}

\begin{document}

\begin{titlepage}
    \maketitle
\end{titlepage}

\tableofcontents

\newpage\section{Άσκηση 1}
Η $ X(z) = \frac{4-\frac{7}{4}z^{-1}+\frac{1}{4}z^{-2}}{1-\frac{3}{4}z^{-1}+\frac{1}{8}z^{-2}} $ γράφεται ως $ X(z) = \frac{32z^2-14z+2}{8z^2-6z+1} $, μετά από πράξεις:
\begin{equation*}\label{eq1}
    \begin{aligned}
        X(z)           & = \frac{4-\frac{7}{4}z^{-1}+\frac{1}{4}z^{-2}}{1-\frac{3}{4}z^{-1}+\frac{1}{8}z^{-2}} & \implies                                                \\
        X(z)           & = \frac{4-\frac{7}{4z}+\frac{1}{4z^2}}{1-\frac{3}{4z}+\frac{1}{8z^2}}                 & \implies                                                \\
        X(z)           & = \frac{4-\frac{7}{4z}+\frac{1}{4z^2}}{1-\frac{3}{4z}+\frac{1}{8z^2}}                 & \overset{z\neq0}{\implies}                              \\
        \frac{X(z)}{z} & = \frac{\frac{4-\frac{7}{4z}+\frac{1}{4z^2}}{1-\frac{3}{4z}+\frac{1}{8z^2}}}{z}       & \overset{\frac{\frac{b}{c}}{a}=\frac{b}{c*a}}{\implies} \\
        \frac{X(z)}{z} & = \frac{4-\frac{7}{4z}+\frac{1}{4z^2}}{(1-\frac{3}{4z}+\frac{1}{8z^2})z}              & \implies                                                \\
        \frac{X(z)}{z} & = \frac{4-\frac{7}{4z}+\frac{1}{4z^2}}{(\frac{8z^2-6z+1}{8z^2})z}                     & \implies                                                \\
        \frac{X(z)}{z} & = \frac{\frac{16z^2-7z+1}{4z^2}}{(\frac{8z^2-6z+1}{8z^2})z}                           & \overset{\frac{\frac{b}{c}}{a}=\frac{b}{c*a}}{\implies} \\
        \frac{X(z)}{z} & = \frac{16z^2-7z+1}{4z^3(\frac{8z^2-6z+1}{8z^2})}                                     & \implies                                                \\
        \frac{X(z)}{z} & = \frac{16z^2-7z+1}{z(\frac{8z^2-6z+1}{2})}                                           & \implies                                                \\
        X(z)           & = \frac{z(16z^2-7z+1)}{z(\frac{8z^2-6z+1}{2})}                                        & \implies                                                \\
        X(z)           & = \frac{16z^2-7z+1}{\frac{8z^2-6z+1}{2}}                                              & \implies                                                \\
        X(z)           & = \frac{32z^2-14z+2}{8z^2-6z+1}                                                                                                                 \\
    \end{aligned}
\end{equation*}

\newpage\section{Άσκηση 2}



\end{document}
